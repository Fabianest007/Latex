\documentclass[spanish]{udpreport}
\usepackage[utf8]{inputenc}
\usepackage[spanish]{babel}
\usepackage[]{float}

\title{Informe Nº1 Redes De Datos}

\author{Javiera Araya,
       Benjamin Morales,
        Fabian Estefania,
        Nicolas Pino.\\
        Profesor:Jaime Álvarez.\\
        Ayudante:Maximiliano Vega.} 

\begin{document}
\maketitle

\tableofcontents
\chapter{Resumen  }

En esta primera experiencia de laboratorio se pone énfasis en lograr reconocer la topología del laboratorio de informática, para ello se tuvo que aplicar inicialmente un análisis a cada equipo conectado en el laboratorio, esto quiere decir, obtener su IP, MAC y ver la categoría del cable, todo con el fin de poder reconocer a todos y cada uno de los “nodos” que componen la red, junto con esto se esquematizo manualmente y de forma ordenada cada uno de estos nodos. 
Teniendo lo primero esquematizado ordenadamente y teniendo más menos clara la topología del lugar, se procede a ver equipo por equipo que lugar del  único Switch disponible en la sala  le corresponde, para ello vamos  por la desconexión de cada cable red y se verifica que luces  se apagan a medida que se va realizando la desconexión. 
Así mismo, junto con localizar el Switch del laboratorio se da también con la ubicación con el Patch Panel el cual va conectado al Switch por cable cruzado. Cabe recordar igualmente que los equipos están conectados al Patch Panel por cable directo.


\chapter{Introducción }



En el presente informe se darán a conocer los aspectos fundamentales de tener conocimiento sobre las redes, para ello esta experiencia de basa en  realizar un estudio completo a la red del laboratorio de informática esto vale decir, poder identificar qué tipo de topología tienen y también poder reconocer los elementos por los cuales está compuesta dicha red por medio de la implementación que hagamos de los recursos que se tienen, en síntesis todo lo realizado nos ayudara a poder una mejor comprensión de cómo se compone la red del laboratorio. 

La importancia de saber la topología de la red es de vital importancia ya que es el paso inicial para ver cómo están los nodos conectados y poder discriminar cuales en conjunto que tipo de elementos existen en el espacio a evaluar.


\chapter{Desarrollo}

\section{Recoleccion de direccines IP}

Para comenzar se obtuvieron las direcciones IP de cada computador perteneciente a la red del laboratorio. Como cada uno utiliza el sistema operativo LINUX, se abre un terminal y se ejecuta el comando "ifconfig", el cuál nos despliega información detallada de la conexión actual del PC, incluida su dirrección IP, realizamos este proceso en todos y cada uno de los equipos de la red.\\

A continuación se presenta una tabla con los datos obtenidos:
\vspace{2cm}
\begin{table}[htbp]
\begin{center}
\begin{tabular}{|l|l|}
\hline
PC & IP \\
\hline \hline
PC1 & 172.16.32.101 \\ \hline
PC2 & 172.16.32.103 \\ \hline
PC3 & 172.16.32.102 \\ \hline
PC4 & 172.16.32.105 \\ \hline
PC5 & 172.16.32.104 \\ \hline
PC6 & 172.16.32.107 \\ \hline
PC7 & 172.16.32.106 \\ \hline
PC8 & 172.16.32.117 \\ \hline
PC9 & 172.16.32.118 \\ \hline
PC10 & 172.16.32.116 \\ \hline
PC11 & 172.16.32.115 \\ \hline
PC12 & 172.16.32.113 \\ \hline
PC13 & 172.16.32.114 \\ \hline
PC14 & 172.16.32.112 \\ \hline
PC15 & 172.16.32.113 \\ \hline
PC16 & 172.16.32.125 \\ \hline
PC17 & 172.16.32.109 \\ \hline
PC18 & 172.16.32.108 \\ \hline
\end{tabular}
\label{tabla:sencilla}
\end{center}
\end{table}
\vspace{14cm}

\section{Recoleccion de direcciones MAC}
Para recolectar las direcciones MAC de todos los equipos presentes en la sala procedimos a acceder a la terminal de linux, mediante el atajo de teclado CTR+ALT+T, aplicamos el comando ifconfig en la terminal,la cual nos proporciona la dirección MAC, IP, entre otros datos, este procedimiento fue realizado al mismo tiempo que se recolectaban IP, esto se realizo en todos y cada uno de los PC
\vspace{2cm}
\begin{table}[H]
\begin{center}
\begin{tabular}{|l|l|l|}
\hline
PC & IP & MAC \\
\hline \hline
PC1 & 172.16.32.101  & 40:A8:F0:51:DB:4C \\ \hline
PC2 & 172.16.32.103  & 40:A8:F0:51:DD:F2 \\ \hline
PC3 & 172.16.32.102  & 40:A8:F0:56:47:8B \\ \hline
PC4 & 172.16.32.105  & 40:A8:F0:51:DB:4E \\ \hline
PC5 & 172.16.32.104  & 40:A8:F0:4E:E8:DD \\ \hline
PC6 & 172.16.32.107  & 40:A8:F0:4D:F7:29 \\ \hline
PC7 & 172.16.32.106  & 40:A8:F0:4E:E8:E3 \\ \hline
PC8 & 172.16.32.117  & 40:A8:F0:53:4F:27 \\ \hline
PC9 & 172.16.32.118  & 40:A8:F0:56:49:3B \\ \hline
PC10 & 172.16.32.116 & 40:A8:F0:53:4E:8C \\ \hline
PC11 & 172.16.32.115 & 40:A8:F0:4E:E8:E2 \\ \hline
PC12 & 172.16.32.113 & 40:A8:F0:56:84:5B \\ \hline
PC13 & 172.16.32.114 & 40:A8:F0:53:4E:9F \\ \hline
PC14 & 172.16.32.112 & 40:A8:F0:51:DC:E1 \\ \hline
PC15 & 172.16.32.113 & 40:A8:F0:4D:F9:E8 \\ \hline
PC16 & 172.16.32.125 & 40:A8:F0:53:4E:96 \\ \hline
PC17 & 172.16.32.109 & 40:A8:F0:56:84:66 \\ \hline
PC18 & 172.16.32.108 & 40:A8:F0:53:4E:93 \\ \hline
\end{tabular}
\label{tabla:sencilla}
\end{center}
\end{table}
\vspace{16cm}

\section{Deteccion de categoria de cables LAN}

luego de recolectar direcciones IP y las MAC de cada computador del laboratorio de informática, se procede a revisar todos y cada uno de los cables de red los cuales cada equipo está conectado, con ello se procede a identificar individualmente cada cable iniciando con la desconexión de este ultimo de la roseta del equipo para así poder tener mejor visión de la descripción que aparece en el cable y así poder observar detenidamente y anotar correctamente la categoría de cada cable, para llevar a cabo el registro, se realiza un esquema de la sala en donde ordenadamente se anota las categorías. Con lo anterior realizado se obtuvo la siguiente tabla en que se anotaron los registros de las categorías de cada cable.

\vspace{2cm}

\begin{table}[H]
\begin{center}
\begin{tabular}{|l|l|l|l|}
\hline
PC & IP & MAC & LAN CAT \\
\hline \hline
PC1 & 172.16.32.101  & 40:A8:F0:51:DB:4C & Cat 5e\\ \hline
PC2 & 172.16.32.103  & 40:A8:F0:51:DD:F2 & Cat 5e\\ \hline
PC3 & 172.16.32.102  & 40:A8:F0:56:47:8B & Cat 5e\\ \hline
PC4 & 172.16.32.105  & 40:A8:F0:51:DB:4E & Cat 5e\\ \hline
PC5 & 172.16.32.104  & 40:A8:F0:4E:E8:DD & Cat 5e\\ \hline
PC6 & 172.16.32.107  & 40:A8:F0:4D:F7:29 & Cat 5e\\ \hline
PC7 & 172.16.32.106  & 40:A8:F0:4E:E8:E3 & Cat 5e\\ \hline
PC8 & 172.16.32.117  & 40:A8:F0:53:4F:27 & Cat 5e\\ \hline
PC9 & 172.16.32.118  & 40:A8:F0:56:49:3B & Cat 5e\\ \hline
PC10 & 172.16.32.116 & 40:A8:F0:53:4E:8C & Cat 5e\\ \hline
PC11 & 172.16.32.115 & 40:A8:F0:4E:E8:E2 & Cat 5e\\ \hline
PC12 & 172.16.32.113 & 40:A8:F0:56:84:5B & Cat 5e\\ \hline
PC13 & 172.16.32.114 & 40:A8:F0:53:4E:9F & Cat 5e\\ \hline
PC14 & 172.16.32.112 & 40:A8:F0:51:DC:E1 & Cat 5e\\ \hline
PC15 & 172.16.32.111 & 40:A8:F0:4D:F9:E8 & Cat 5\\ \hline
PC16 & 172.16.32.125 & 40:A8:F0:53:4E:96 & Cat 5e\\ \hline
PC17 & 172.16.32.109 & 40:A8:F0:56:84:66 & Cat 5e\\ \hline
PC18 & 172.16.32.108 & 40:A8:F0:53:4E:93 & Cat 5e\\ \hline
\end{tabular}
\label{tabla:sencilla}
\end{center}
\end{table}

\vspace{14cm}
\section{Deteccion de puertos en Pach panel y Switch}

Luego de obtener toda la informacion sobre los Pcs, tocaba ver como estaba constituido el cableado y en que puertos del switch estaba conectado cada equipo, para esto el procedimiento que se siguio fue desconectar un Pc de la red y verificar cual de los puertos del switch dejaba de existir algun tipo de actividad, de esta manera logramos identificar todos y cada unos de los puertos de switch.\\

El siguiente paso fue identificar los puertos conectados en el patch panel, para esto se siguio un procedimiento parecido al anterio en el que se desconecto cada uno de los puertos del patch panel, luego se verificaba cual puerto del switch dejaba de tener actividad, asi se logro identificar desde que Pc provenia el cable y hacia cual puerto del switch se dirigia.\\

Luego de todo el procedimiento obtuvimos la siguiente tabla:

\vspace{2cm}

\begin{table}[htbp]
\begin{center}
\begin{tabular}{|l|l|l|l|l|l|}
\hline
PC & IP & MAC & LAN CAT & PATCH & SWITCH \\
\hline \hline
PC1 & 172.16.32.101  & 40:A8:F0:51:DB:4C & Cat 5e & 2  & 1  \\ \hline
PC2 & 172.16.32.103  & 40:A8:F0:51:DD:F2 & Cat 5e & 3  & 3  \\ \hline
PC3 & 172.16.32.102  & 40:A8:F0:56:47:8B & Cat 5e & 4  & 4  \\ \hline
PC4 & 172.16.32.105  & 40:A8:F0:51:DB:4E & Cat 5e & 5  & 5  \\ \hline
PC5 & 172.16.32.104  & 40:A8:F0:4E:E8:DD & Cat 5e & 6  & 8  \\ \hline
PC6 & 172.16.32.107  & 40:A8:F0:4D:F7:29 & Cat 5e & 7  & 7  \\ \hline
PC7 & 172.16.32.106  & 40:A8:F0:4E:E8:E3 & Cat 5e & 8  & 6  \\ \hline
PC8 & 172.16.32.117  & 40:A8:F0:53:4F:27 & Cat 5e & 9  & 9  \\ \hline
PC9 & 172.16.32.118  & 40:A8:F0:56:49:3B & Cat 5e & 10 & 10 \\ \hline
PC10 & 172.16.32.116 & 40:A8:F0:53:4E:8C & Cat 5e & 11 & 12 \\ \hline
PC11 & 172.16.32.115 & 40:A8:F0:4E:E8:E2 & Cat 5e & 12 & 11 \\ \hline
PC12 & 172.16.32.113 & 40:A8:F0:56:84:5B & Cat 5e & 13 & 13 \\ \hline
PC13 & 172.16.32.114 & 40:A8:F0:53:4E:9F & Cat 5e & 14 & 14 \\ \hline
PC14 & 172.16.32.112 & 40:A8:F0:51:DC:E1 & Cat 5e & 15 & 15 \\ \hline
PC15 & 172.16.32.111 & 40:A8:F0:4D:F9:E8 & Cat 5  & 16 & 16 \\ \hline
PC16 & 172.16.32.125 & 40:A8:F0:53:4E:96 & Cat 5e & 17 & 18 \\ \hline
PC17 & 172.16.32.109 & 40:A8:F0:56:84:66 & Cat 5e & 18 & 17 \\ \hline
PC18 & 172.16.32.108 & 40:A8:F0:53:4E:93 & Cat 5e & 19 & 19 \\ \hline
\end{tabular}
\label{tabla:sencilla}
\end{center}
\end{table}

\chapter{Grafico de la red}
\begin{figure}[H]
\centering
\includegraphics[scale=0.4]{RedUDP}
\end{figure}

\chapter{Conclusión}

Al realizar esta activdad se logró descubrir como esta distribuida la red en el laboratorio de informatica de la universidad diego portales y concluir que tiene una topologia de tipo estrella en la cual el switch seria el centro y los equipos serian los extemos, también se determino la velocidad maxima de conexion al saber que se utilizan los cables de red de categoria 5e, por tanto la red podria llegar a una velocidad de hasta 1GB/s.

\end{document}
